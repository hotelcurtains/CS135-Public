\documentclass{article}
\usepackage{amsmath}
\usepackage{mathbbol}
\usepackage{mathtools}
\usepackage[letterpaper,top=1in,bottom=1in,left=1in,right=1in]{geometry}
\usepackage{chngcntr}
\usepackage{amssymb}
\usepackage[verbose]{placeins}
\counterwithin*{equation}{section}
\counterwithin*{equation}{subsection}
\renewcommand{\thesubsection}{\thesection.\alph{subsection}}

\title{Written Assignment 4}
\author{CS135-B/LF}
\date{February 25, 2024}

\begin{document}

\maketitle

\section{}
\subsection{}
32 is not a square square-free number. It is divisible by 4, which is $ 2^2 $. 

\subsection{}
This is an invalid way to use the superset operation. John is trying to compare an integer with a set, which is not possible. He should have written:
\begin{equation*}
P (\{17, 19, 22, 26, 30\}, \{30\})
\end{equation*}


\section{}
\subsection{}
\begin{itemize}
\item Area 1 = $ \overline{A} \cap \overline{B} \cap C  $
\item Area 2 = $ \overline{A} \cap B \cap \overline{C}  $ 	
\item Area 3 = $ A \cap \overline{B} \cap \overline{C}  $
\item Area 4 = $ \overline{A} \cap \overline{B} \cap \overline{C}$
\item Area 5 = $ \overline{A} \cap B \cap C  $ 
\item Area 6 = $ A \cap \overline{B} \cap C  $
\item Area 7 = $ A \cap B \cap \overline{C}  $
\item Area 8 = $ A \cap B \cap C $
\end{itemize}

\subsection{}
There would be 16 areas. The amount of combinations of $A, B, C,$ and $D$ would be the length of the power set $P ( \{ A, B, C, D \} )$. The power set of a set with 4 elements has $2^4 = 16$ elements.

\subsection{}
The simple way is to find all combinations of $A \cap B \cap C \cap D$ possible with 0-4 of the sets replaced with its compliment.

\section{}
\subsection{}
\begin{equation*}
    S(\{ \sqrt{2}, \sqrt{3}, \sqrt{5} \}) = 
    \{ \sqrt{2}, \sqrt{3}, \sqrt{5} , \{ \sqrt{2}, \sqrt{3}, \sqrt{5} \} \}
\end{equation*}

\subsection{}

\begin{align*}
    S(\emptyset ) & = \emptyset \cup \{ \emptyset \} \\
    & = \{ \emptyset \} \\ \\
    S(S(\emptyset)) &= \{ \emptyset \} \cup \{ \{ \emptyset \} \} \\
    &= \{ \emptyset, \{ \emptyset \} \} \\ \\
    S(S(S(\emptyset))) & = \{ \emptyset, \{ \emptyset \} \}  \cup \{ \{ \emptyset, \{ \emptyset \} \} \} \\
    & = \{ \emptyset , \{ \emptyset \} , \{ \emptyset , \{ \emptyset \} \} \} \\
    & = \{ \{ \} , \{ \{ \} \} , \{ \{ \} , \{ \{ \} \} \} \}
\end{align*}

\subsection{} 
$|S(P)| = |P| +1$. For instance, if $P = \{ 5, 6, 7 \}$, then $S(P) = \{ 5, 6, 7, \{ 5, 6, 7 \} \}$. 

\section{}
\begin{align*}
    & (A \setminus B ) \setminus C & \text{Original statement} & \\
    & (A \cap \overline{B} ) \setminus C & \text{Set Difference Law} & \\
    & (A \cap \overline{B} ) \cap \overline{C} & \text{Set Difference Identity} & \\
    & A \cap (\overline{B}  \cap \overline{C}) & \text{Associativity} & \\
    & A \setminus \overline{(\overline{B}  \cap \overline{C})} & \text{Set Difference Identity} & \\
    & A \setminus (\overline{\overline{B}}  \cup \overline{\overline{C}}) & \text{De Morgan's Law} & \\
    & A \setminus (B  \cup \overline{\overline{C}}) & \text{Complementation Law} & \\
    & A \setminus (B  \cup C) & \text{Complementation Law} &
\end{align*}


\end{document}