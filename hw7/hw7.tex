\documentclass{article}
\usepackage{amsmath}
\usepackage{amsfonts}
\usepackage{mathbbol}
\usepackage{mathtools}
\usepackage[letterpaper,top=1in,bottom=1in,left=1in,right=1in]{geometry}
\usepackage{chngcntr}
\usepackage{amssymb}
\usepackage[verbose]{placeins}
\counterwithin*{equation}{section}
\counterwithin*{equation}{subsection}
\renewcommand{\thesubsection}{\thesection.\alph{subsection}}

\DeclarePairedDelimiter{\floor}{\lfloor}{\rfloor}
\DeclarePairedDelimiter{\ceil}{\lceil}{\rceil}

\renewcommand{\mod}{\text{ mod }}

\title{Written Assignment 7}
\author{CS 135-B/LF}
\date{April 4, 2024}

\begin{document}
\maketitle
\raggedright

\section{}
\subsection{}
\begin{align*}
    T(1) &= 2T(\floor{\frac{1}{4}}) + \sqrt{1} \\
         &= 2T(0) + 1 \\
         &= 2(1) + 1 \\
    T(1) &= 3
\end{align*}
\begin{align*}
    T(2) &= 2T(\floor{\frac{2}{4}}) + \sqrt{2} \\
         &= 2T(0) + \sqrt{2} \\
    T(2) &= 2 + \sqrt{2}\\
         &\approx 3.41421
\end{align*}

\subsection{}
We need to prove that $T(n) \leq 3\sqrt{n}\log_2 n$ for all $n\geq 2$.\\
Let's check this premise for basis $n=2$:
\begin{align*}
    T(2) &\leq 3\sqrt{2}\log_2(2) \\
    2 + \sqrt{2} &\leq 3\sqrt{2} \\
    2 &\leq 2\sqrt{2} \\
    1 &\leq \sqrt{2} \\
    &\top
\end{align*}
Inductive hypothesis: Let $k$ be an integer $\geq 2$. We'll assume that $T(i) \leq 3\sqrt{i}\log_2 i$ for all $2 \leq i\leq k, i\in \mathbb{Z}$.\\
Now we must prove that
\begin{equation*}
    T(k+1) \leq 3\sqrt{k+1}\log_2(k+1).
\end{equation*}
\begin{align*}
    T(k+1) &= 2T(\floor*{\frac{k+1}{4}}) + \sqrt{k+1} & \text{By definition of $T$}\\
           &\leq 2(3\sqrt{\floor*{\frac{k+1}{4}}}\log_2\floor*{\frac{k+1}{4}}) + \sqrt{k+1} & \text{By the inductive hypothesis} \\
           &\leq 2(3\sqrt{\frac{k+1}{4}}\log_2(\frac{k+1}{4})) + \sqrt{k+1} & \floor{x} \leq x \\
           &= 2(\frac{3}{2}\sqrt{k+1}\log_2(\frac{k+1}{4})) + \sqrt{k+1} & \text{By algebra} \\
           &= 3\sqrt{k+1}\log_2(\frac{k+1}{4}) + \sqrt{k+1} \\
           &= 3\sqrt{k+1}(\log_2(k+1) - \log_2(4)) + \sqrt{k+1} \\
           &= 3\sqrt{k+1}(\log_2(k+1) - 2) + \sqrt{k+1} \\
           &= 3\sqrt{k+1}\log_2(k+1) - 6\sqrt{k+1} + \sqrt{k+1} \\
           &= 3\sqrt{k+1}\log_2(k+1) - 5\sqrt{k+1} \\
           &< 3\sqrt{k+1}\log_2(k+1)                                    &\blacksquare
\end{align*}

\section{}
Let $a,b,c,d,m \in \mathbb{Z}$ with $m \geq 2$. We'll prove that $a \equiv b (\text{mod } m) \wedge c \equiv d (\text{mod } m) \implies a-c\equiv b-d (\text{mod } m)$. Given


\begin{equation*}
    \begin{cases}
            a \equiv b (\text{mod } m) \\
            c \equiv d (\text{mod } m)
    \end{cases}
\end{equation*}

we can break this down into
\begin{equation*}
    \begin{cases}
        a \text{ mod } m = b \text{ mod } m \\
        c \text{ mod } m = d \text{ mod } m
    \end{cases}
\end{equation*}
by the definition of congruence. We'll scale the second equation by $-1$,
\begin{equation*}
    \begin{cases}
    a \text{ mod } m &= b \text{ mod } m \\
    (-c) \text{ mod } m &= (-d) \text{ mod } m
\end{cases}
\end{equation*}

add the equations together as one can do with a system,
\begin{equation*}
    (a-c)\text{ mod }m = (b-d)\text{ mod }m
\end{equation*}

and rewrite this as the congruence
\begin{equation*}
    a-c\equiv b-d (\text{mod } m).
\end{equation*}
$\blacksquare$

\section{}
No. For $k=6$, $2*3*5*7*11*13 +1 = 30031$.
30031 has factors $\{59, 509\}$ which means 30031 is not prime.
Therefore $p_1p_2\cdots p_k +1$ is not always prime.

\section{}
No. Since 30 is not divisible by 29, let $m=30$:
\begin{align*}
    & 2m^2 + 29 \\
    &= 2(30)^2 + 29 \\
    &= 1829
\end{align*}
1829 has factors $\{ 31, 59 \}$ which means 1829 is not prime. Therefore $2m^2 + 29$ does not always produce a prime number when m is not divisible by 29.


\section{}
Let's prove a basis where $k=1$: and since $1^4 -1 = 0$ and 0 is divisible by 16, this basis holds.\\
Let's assume that $k$ is divisible by 16. With this we can say that there is some integer $s$ for which $k^4 - 1 = 16s$.\\
Let's prove that $k+2$ is divisible by 16 ($k+2$ because k and this equation's result must be odd). We can do this by proving that there is some integer which, when multiplied by 16, gives some expression of $k$. For some integer $t$:
\begin{align*}
            16t &= (k+2)^4 - 1 \\
                &= (k^2 + 4k + 4)^2 - 1 \\
                &= k^4 + 8k^3 + 24k^2 + 32k + 16 - 1 \\
                &= k^4 - 1 + 8k^3 + 24k^2 + 32k + 16 \\
\intertext{By the inductive hypothesis: there is some integer $s$ for which $k^4 - 1 = 16s$}
                &= 16s + 8k^3 + 24k^2 + 32k + 16 \\
                &= 16s + 8k^2(k+3) + 16(2k+1)
\end{align*}
Since $k$ is an integer, we can reasonably expect $2k+1$ to be an integer as well. $8k^2(k+3)$ will be a multiple of 8 and some odd number $k^2(k+3)$, which is guaranteed to be divisible by 16. We can think of this equation as
\begin{equation*}
    (\text{Some multiple of 16}) = (\text{The sum of multiples of 16})
\end{equation*}
which proves that $k+2$ produces a multiple of 16, which proves that for all odd integers $k \geq 1$, $k^4 - 1$ is divisible by 16.


\section{}
\begin{align*}
    &  10^{6,000,000,000,000,000,000,000,000,000,000,000,000,000,002} \mod 13 \\
    &= (10 \mod 13)(10 \mod 13)(10^{6,000,000,000,000,000,000,000,000,000,000,000,000,000,000} \mod 13) \mod 13 \\
    &= 10 \cdot 10 \cdot (10^{6\times 10^{42}} \mod 13) \mod 13 \\
    &= 100 \cdot (10^6 \mod 13) \cdot (10^{10^{42}} \mod 13) \mod 13 \\
    &= 100 \cdot 1 \cdot (10^{10^{21}} \mod 13)^2 \mod 13\\
    &= 100 \cdot ((10^{10^7} \mod 13)^3 \mod 13)^2 \mod 13\\
    &= 100 \cdot (((10^{10} \mod 13)^7 \mod 13))^3 \mod 13)^2 \mod 13\\
    \intertext{Now we have a lot of easily-calculable modulus operations.}
    &= 100 \cdot ((3^7 \mod 13)^3 \mod 13)^2 \mod 13\\
    &= 100 \cdot 1^2 \mod 13\\
    &= 100 \cdot 1 \mod 13\\
    &= 100 \mod 13 \\
    &= 9
\end{align*}
\end{document}