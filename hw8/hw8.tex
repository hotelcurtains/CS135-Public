\documentclass{article}
\usepackage{amsmath}
\usepackage{amsfonts}
\usepackage{mathbbol}
\usepackage{mathtools}
\usepackage[letterpaper,top=1in,bottom=1in,left=1in,right=1in]{geometry}
\usepackage{chngcntr}
\usepackage{amssymb}
\usepackage[verbose]{placeins}

\counterwithin*{equation}{section}
\counterwithin*{equation}{subsection}
\renewcommand{\thesubsection}{\thesection.\alph{subsection}}

\DeclarePairedDelimiter{\floor}{\lfloor}{\rfloor}
\DeclarePairedDelimiter{\ceil}{\lceil}{\rceil}
\renewcommand{\mod}{\text{ mod }}

\title{Written Assignment 8}
\author{CS 135-B/LF}
\date{April 13, 2024}

\begin{document}
\maketitle
\raggedright

\section{}
Using the division algorithm, we can gather that all integers can be represented by some multiple of 6 plus some integer $0 \leq q \leq 5$. We can denote this as $6k + q$. If $q$ is equal to 0, 2, or 4, the sum $6k + q$ will be even, and therefore it cannot represent the primes $p$ or $p+2$. If $q$ is equal to 3 then $6k + q$ will be divisible by 3 and again, cannot represent $p$ or $p+2$. This leaves only the options $q = 1$ and $q = 5$. Again using the division algorithm we understand that the case where q = 5, $6k + 5$, can be written as $6k - 1$. As such, for any prime $p = 6k-1$, we can also represent the prime $p+2$ as $6k+1$. $\blacksquare$
% http://edshare.soton.ac.uk/2379/4/MA309ex2_4.pdf


\section{}
\subsection{}
To find the inverse of $20 \mod 1343$, we need to find $20x \equiv 1 (\mod 1343)$:
\begin{table}[hbt!]
    \begin{tabular}{c|cccc}
    Line & Q  & R    & x    & y   \\ \hline
    1    &    & 20   & 1    & 0   \\
    2    &    & 1343 & 0    & 1   \\
    3    & 0  & 20   & 1    & 0   \\
    4    & 67 & 3    & -67  & 1   \\
    5    & 6  & 2    & 403  & -6  \\
    6    & 1  & 1    & -470 & 7   \\
    7    & 2  & 0    & 1343 & -20
    \end{tabular}
\end{table}
% https://www.tablesgenerator.com/

Using line 6, we know that $-470 \cdot 20 + 7 \cdot 1343 $ and as such the multiplicative inverse for $20 \mod 1343$ is $-470$. If we need it to be positive, we can add $1343$ to $-470$ and retrieve 873 because $-470 \equiv 873 (\mod 1343)$.

\subsection{}
Using multiplicative inverse $20 \mod 1343$ we can gather that one possible $x= 7\cdot 873 = 6111$ so $20 \cdot 6111 \equiv 122220 \equiv 7(\mod 1343)$. Generally, $x \equiv 122220 (\mod 1343)$.


\section{}
\subsection{}
Find $\phi(100)$.\\
We can find that $100 = 2^2 \times 5^2$. Therefore 100 has prime divisors 2 and 5.\\
Fact from Lecture 20: For prime divisors of $n$ $p_1,p_2,\dots,p_k$, $\phi(n) = n(1-1/p_1)(1-1/p_2)\cdots(1-1/p_k)$\\
This gives us $\phi(100) = 100(1-1/2)(1-1/5)$ which works out to 40.

\subsection{}
We can use $\mod 100$ to find the last two digits of any number. Thus we now must solve $17^{800,000,000,000,000,000,000,000,000,000,000,001} \mod 100$. \\
To make sure we can use Euler's Theorem, we can check that gcd(17,100) = 1. Since 17 is prime this statement is clearly true.\\
We know by 3.a that $\phi(100)=40$, and using Euler's theorem we can deduce that $17^{\phi(100)} \equiv 17^{40} \equiv 1 (\mod 100)$. We can break $17^{800,000,000,000,000,000,000,000,000,000,000,001}$ into 
$17^1 \cdot 17^{8 \cdot 10^{35}}$, then 
$17^1 \cdot 17^{80 \cdot 10^{34}}$, then 
$17^1 \cdot 17^{40^{2 \cdot 10^{34}}}$.
Then we can deduce that $17 * (17^{40})^{10^{34}} \equiv 17 * 1^{10^{34}} \equiv 17 (\mod 100)$. This can be rewritten as $17^{800,000,000,000,000,000,000,000,000,000,000,001} \equiv 17 (\mod 100)$. Therefore, the last two digits of $17^{800,000,000,000,000,000,000,000,000,000,000,001}$ are 17.


\section{}
% https://www.youtube.com/watch?v=2-tdwLqyaKo
We are given the following:
\begin{gather*}
    x \equiv 8 \mod 11 \\
    x \equiv 5 \mod 17 \\
    x \equiv 16 \mod 29 \\
    x \equiv 24 \mod 31
\end{gather*}
Along with the information that $x$ has no more than 5 digits. Since none of 11, 17, 29, and 31 have any common factors, the computer scientist son might have realized that this is a setup for the Chinese Remainder Theorem. We'll start by gathering:
\begin{align*}
    m &=   11 \cdot 17 \cdot 29 \cdot 31 = 168113\\
    M_1 &= \ \ \ \ \ 17 \cdot 29 \cdot 31 = 15283\\
    M_2 &= 11 \cdot \ \ \ \ \ 29 \cdot 31 = 9889\\
    M_3 &= 11 \cdot 17 \cdot \ \ \ \ \ 31 = 5797\\
    M_4 &= 11 \cdot 17 \cdot 29 \ \ \ \ \ = 5423
\end{align*}
Now we need to find the multiplicative inverses of
\begin{align*}
    M_1 &\equiv 15283 (\mod 11) \\
    M_2 &\equiv 9889 (\mod 17) \\
    M_3 &\equiv 5797 (\mod 29) \\
    M_4 &\equiv 5423 (\mod 31).
\end{align*}
For each $M_k$ we'll call its multiplicative inverse $y_k$. Using the Euclidean algorithm, we can calculate that 
\begin{align*}
    y_1 &= 3 \\
    y_2 &= 10 \\
    y_3 &= 19 \\
    y_4 &= 15
\end{align*}
Now we can find our solution by calculating
\begin{gather*}
    8 \cdot 15283 \cdot 3 \\
    +\ 5 \cdot 9889 \cdot 10 \\
    +\ 16 \cdot 5797 \cdot 19 \\
    +\ 24 \cdot 5423 \cdot 15 \\
    = 4575810
\end{gather*}
and taking 4575810 mod 168113 which is equal to 36759. With a quick calculation we can find that 36759 matches all of the requirements to be congruent to x in all given cases. Since this number also has 5 digits, it must be the vault combination.


\section{}
% https://www.cip.ifi.lmu.de/~grinberg/t/21fin/lec22.pdf
We are given positive integers $a, b$ and an integer $c$ where $a \mid c$, $b \mid c$ and gcd$(a,b) = 1$. \\
According to Bezout's theorem, for positive integers $a$ and $b$, there are integers $x$ and $y$ where
\begin{equation*}
    \text{gcd}(a,b) = xa+yb
\end{equation*}
Since $b\mid c$, it follows that $ab\mid ac$ and $ac \mid xac$. 
Since $a\mid c$, it follows that $ab\mid cb$ and $cb \mid ycb$. 
By the division algorithm, $ab\mid xac$ and $ab\mid ycb$. 
We can sum these multiples of $ab$ and still have a multiple of $ab$, therefore $ab\mid (xac+ycb) = c(xa+yb)$. 
By Bezout's theorem, $xa+yb = \text{gcd}(a,b) = 1$ which is given. 
Now we know that $ab\mid xac+ycb \implies ab\mid c(1) \implies ab\mid c$ which was to be proven. $\blacksquare$


\section{}
Let's take $a=2$, $b=2$, and $c=2$. gcd(2, 2) $= 2 >1$, $2\mid 2 \implies a\mid c$ and $b\mid c$, therefore the premises are true. However, $ab = 2 \times 2 = 4$ and $4 \nmid 2 \implies ab \nmid c$. $\blacksquare$

\end{document}